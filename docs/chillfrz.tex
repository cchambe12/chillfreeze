\documentclass{article}\usepackage[]{graphicx}\usepackage[]{color}
% maxwidth is the original width if it is less than linewidth
% otherwise use linewidth (to make sure the graphics do not exceed the margin)
\makeatletter
\def\maxwidth{ %
  \ifdim\Gin@nat@width>\linewidth
    \linewidth
  \else
    \Gin@nat@width
  \fi
}
\makeatother

\definecolor{fgcolor}{rgb}{0.345, 0.345, 0.345}
\newcommand{\hlnum}[1]{\textcolor[rgb]{0.686,0.059,0.569}{#1}}%
\newcommand{\hlstr}[1]{\textcolor[rgb]{0.192,0.494,0.8}{#1}}%
\newcommand{\hlcom}[1]{\textcolor[rgb]{0.678,0.584,0.686}{\textit{#1}}}%
\newcommand{\hlopt}[1]{\textcolor[rgb]{0,0,0}{#1}}%
\newcommand{\hlstd}[1]{\textcolor[rgb]{0.345,0.345,0.345}{#1}}%
\newcommand{\hlkwa}[1]{\textcolor[rgb]{0.161,0.373,0.58}{\textbf{#1}}}%
\newcommand{\hlkwb}[1]{\textcolor[rgb]{0.69,0.353,0.396}{#1}}%
\newcommand{\hlkwc}[1]{\textcolor[rgb]{0.333,0.667,0.333}{#1}}%
\newcommand{\hlkwd}[1]{\textcolor[rgb]{0.737,0.353,0.396}{\textbf{#1}}}%
\let\hlipl\hlkwb

\usepackage{framed}
\makeatletter
\newenvironment{kframe}{%
 \def\at@end@of@kframe{}%
 \ifinner\ifhmode%
  \def\at@end@of@kframe{\end{minipage}}%
  \begin{minipage}{\columnwidth}%
 \fi\fi%
 \def\FrameCommand##1{\hskip\@totalleftmargin \hskip-\fboxsep
 \colorbox{shadecolor}{##1}\hskip-\fboxsep
     % There is no \\@totalrightmargin, so:
     \hskip-\linewidth \hskip-\@totalleftmargin \hskip\columnwidth}%
 \MakeFramed {\advance\hsize-\width
   \@totalleftmargin\z@ \linewidth\hsize
   \@setminipage}}%
 {\par\unskip\endMakeFramed%
 \at@end@of@kframe}
\makeatother

\definecolor{shadecolor}{rgb}{.97, .97, .97}
\definecolor{messagecolor}{rgb}{0, 0, 0}
\definecolor{warningcolor}{rgb}{1, 0, 1}
\definecolor{errorcolor}{rgb}{1, 0, 0}
\newenvironment{knitrout}{}{} % an empty environment to be redefined in TeX

\usepackage{alltt}
\usepackage{Sweave}
\usepackage{float}
\usepackage{graphicx}
\usepackage{tabularx}
\usepackage{siunitx}
\usepackage{amssymb} % for math symbols
\usepackage{amsmath} % for aligning equations
\usepackage{textcomp}
\usepackage{mdframed}
\usepackage[small]{caption}
\setlength{\captionmargin}{30pt}
\setlength{\abovecaptionskip}{0pt}
\setlength{\belowcaptionskip}{10pt}
\topmargin -1.5cm        
\oddsidemargin -0.04cm   
\evensidemargin -0.04cm
\textwidth 16.59cm
\textheight 21.94cm 
%\pagestyle{empty} %comment if want page numbers
\parskip 7.2pt
\renewcommand{\baselinestretch}{2}
\parindent 0pt
\usepackage{lineno}
\linenumbers
\usepackage{natbib}
\bibliographystyle{..//references/styles/besjournalsnew.bst}

%cross referencing:
\usepackage{xr}
\usepackage{xr-hyper}
\externaldocument{/Users/CatherineChamberlain/Documents/git/chillfreeze/docs/chillfrz_supp}

\newmdenv[
  topline=true,
  bottomline=true,
  skipabove=\topsep,
  skipbelow=\topsep
]{siderules}
\IfFileExists{upquote.sty}{\usepackage{upquote}}{}
\begin{document}

\noindent \textbf{\Large{False spring damage to temperate tree saplings is amplified with winter warming}}
%\noindent \textbf{\Large{False springs coupled with warming winters amplify temperate tree damage}}

\noindent Authors:\\
C. J. Chamberlain $^{1,2}$, K. Woodruff $^{1}$ \& E. M. Wolkovich $^{1,2,3}$
\vspace{2ex}\\
\emph{Author affiliations:}\\
$^{1}$Arnold Arboretum of Harvard University, 1300 Centre Street, Boston, Massachusetts, USA 02131; \\
$^{2}$Organismic \& Evolutionary Biology, Harvard University, 26 Oxford Street, Cambridge, Massachusetts, USA 02138; \\
$^{3}$Forest \& Conservation Sciences, Faculty of Forestry, University of British Columbia, 2424 Main Mall, Vancouver, BC V6T 1Z4\\
\vspace{2ex}
$^*$Corresponding author: 248.953.0189; cchamberlain@g.harvard.edu\\

\renewcommand{\thetable}{\arabic{table}}
\renewcommand{\thefigure}{\arabic{figure}}
\renewcommand{\labelitemi}{$-$}
\setkeys{Gin}{width=0.8\textwidth}

%%%%%%%%%%%%%%%%%%%%%%%%%%%%%%%%%%%%%%%%%%%%%%%
%%%%%%%%%%%%%%%%%%%%%%%%%%%%%%%%%%%%%%%%%%%%%%%

\section*{Abstract} % NEED TO FIX A LOT!!!
With warming temperatures in the Northern Hemisphere, spring phenology (i.e., budburst and leafout) is advancing. Late spring freezing events that occur after trees initiate budburst---known as false springs---are very damaging and they are predicted to increase in certain regions as climate change progresses. Additionally, over-winter chilling temperatures are predicted to decrease as winter temperatures warm, potentially impacting phenology and, ultimately, growth. Here, we assessed the effects varying durations of over-winter chilling on sapling phenology and growth across eight temperate tree and shrub species. Half of the individuals were then exposed to false spring conditions. We found that false springs impacted phenology, growth and leaf characteristics but not phenological rank within a community. Chilling length greatly influences the critical budburst to leafout phases, and can compensate for the adverse effects of false springs on phenology and risk of multiple frosts. Additionally, false springs impact leaf toughness, thickness and chlorophyll content and consistenly lead to damage to the shoot apical mersitem. We therefore expect climate change to reshape forest communities not through temporal reassembly but rather through impacts on growth and leaf traits from the coupled effects of false springs with decreases in over-winter chilling under future climate change scenarios.

\textbf{Synthesis:} With climate change and warming temperatures, over-winter chilling is anticipated to decrease and false springs are predicted to increase in certain regions. This combination could greatly impact plant performance, survival and shape species distributions, ultimately affecting crucial processes such as carbon uptake and nutrient cycling.

\vspace{2ex}
\textit{Keywords:} false spring, climate change, phenology, spring freeze, forest recruitment, temporal reassembly, budburst, temperate

\section*{Introduction}
The timing of spring in temperate deciduous forests shapes plant and animal communities and influences ecosystem services from agriculture to forest management. With warming temperatures in the Northern Hemisphere, spring phenology (i.e., budburst and leafout, which  are strongly cued by temperature) is advancing causing longer growing seasons \citep{Chuine2001} and reshaping these services. In one major example, advancing spring phenology has lead to increased carbon uptake across temperate forests, which are essential carbon sinks that combat the negative effects of climate change \citep{Keenan2014}. But climate change has other important effects that could cause declines in these services: specifically cold snaps during the spring and reduced cool temperatures in the winter.
  
And though the Northern Hemisphere is getting warmer, climate change is affecting general temperature trends but extreme weather events (e.g., polar vortexes) are still occurring. These weather events can in turn have big impacts on plant development each spring. One such event is known as a `false spring', which is when temperatures drop below freezing \citep[][i.e., below -2.2$^{\circ}$C]{Schwartz2002} after budburst has initiated. Damage from false spring events can have cascading effects to pollinators \citep{Boggs2012, Pardee2017}, nutrient cycling and carbon uptake as well as forest recruitment \citep{Hufkens2012, Klosterman2018, Richardson2013}.

Furthermore, false springs can increase the chance of additional freezes within a growing season by extending the period in which plants are most at risk---the time between budburst and leafout (what we refer to as the `duration of vegetative risk'). Observational studies suggest plants take longer to re-flush leaves after a false spring---up to 38 days \citep{Augspurger2009, Augspurger2013, Gu2008, Menzel2015}, which could lead to additional false springs in a season \citep{Augspurger2009}. False springs are predicted to increase in certain regions as climate change progresses \citep{Ault2015, Liu2018, Zohner2020}, thus understanding the impacts of false spring events on forests is essential for forest management strategies and climate forecasting \citep{OBrien2019}.
  
Warmer winters may also play a critical role in the future of forests as they directly impact one of the  major cues plants use to time budburst: over-winter cold temperatures (chilling), (in addition to warming spring temperatures (forcing) and longer daylengths). Many temperate plants have evolved chilling requirements to avoid leafout during warm snaps in the middle of the winter, but with climate change, chilling requirements may not be met. If chilling is not met, plants may leaf out much slower or incompletely, which can in turn affect freeze tolerance. Thus, understanding the interplay of warming winters and false spring risk is critical to predict how temperate forests will change in the future.
  
This interaction between winter chilling and false springs may vary across species within a community, as species have likely evolved along a trade-off of risking spring freezes for early access to resources. Ideally, all individuals of all species would evolve to require high levels of chilling to delay budburst and ultimately diminish false spring risk but competition for nutrients, water and light resources in the early spring pushes individuals to leafout earlier. Young trees and understory species generally initiate budburst before the canopy trees to benefit from higher light levels \citep {Augspurger2008, Vitasse2013}, which potentially puts these species and individuals at higher risk of freeze damage \citep{Vitasse2014}. Thus, successful forest recruitment requires seedlings and saplings to minimize false spring risk while maximizing growth.
 
The combination of species- and lifestage-level differences in responses to false springs, chilling and climate change could reshape the temporal assembly of forest communities. Species typically leafout in a similar sequence, with understory species leafing out earlier and higher canopy trees leafing out last but many studies are predicting substantial shifts in chronological order and reassembly of species' leafout with climate change \citep{Roberts2015, Laube2014}. As warming alters winter temperatures and false spring prevalence, phenological cues and their interactions are anticipated to change, which could greatly alter competition and recruitment among forest species for early season resources and ultimately impact species diversity and carbon uptake in temperate forests.
  
Here, we assessed the effects of over-winter chilling length and false springs on sapling phenology and growth across eight temperate tree and shrub species. Individuals experienced different levels of over-winter chilling and then half of the individuals were exposed to a false spring event. We then observed individuals for the remainder of the growing season to ask: (1) How does the accumulation of over-winter chilling hours and (2) how do false spring events impact phenology, growth and physical leaf traits?

\section*{Materials and Methods}
\subsection*{Plant Selection and Material}
We used eight temperate woody plant tree and shrub species with varying phenologies, that were not used as crops or ornamental species: \textit{Acer saccharinum} L., \textit{Alnus incana rugosa} L., \textit{Betula papyrifera} Marsh., \textit{Betula populifolia} Marsh., \textit{Cornus racemosa} Lam., \textit{Salix purpurea} L., \textit{Sorbus americana} Marsh., and \textit{Viburnum dentatum} L. Two additional species---\textit{Fagus grandifolia} and \textit{Nyssa sylvatica}---were originally included in the experimental design, but were not delivered in a usable condition and thus could not be used in the experiment. We received 48 dormant, one year old, bare root saplings---each measuring 6-12 inches---for each species from Cold Stream Farm LLC (Freesoil, MI; 44$^{\circ}$6' N -86$^{\circ}$12' W) for a total of 384 individuals. Upon receipt, plants were potted in 656ml deepots with Fafard \#3B Metro Mix soil and placed in growth chambers at the Weld Hill Research Building of the Arnold Arboretum (Boston, MA; 42$^{\circ}$17' N -71$^{\circ}$8' W) at 4$^{\circ}$C to maintain dormancy. After all individuals had leafed out, all saplings were up-potted to 983ml deepots and fertilized with SCOTTS 15-9-12 Osmocote Plus 5-6.

\subsection*{Growth Chamber and Greenhouse Conditions}
Individuals were randomly selected and placed in six experimental treatments: four weeks of chilling at 4$^{\circ}$C x no false spring, four weeks of chilling at 4$^{\circ}$C x false spring, six weeks of chilling at 4$^{\circ}$C x no false spring, six weeks of chilling at 4$^{\circ}$C x false spring, eight weeks of chilling at 4$^{\circ}$C x no false spring, eight weeks of chilling at 4$^{\circ}$C x false spring. While individuals were in the growth chamber under chilling conditions, photoperiod was maintained at eight hour days. Lighting within the chambers was provided through a combination of T5HO fluorescent lamps with halogen incandescent bulbs at roughly 250 $\mu mol/m^{2}/s$. Individuals were rotated within and among growth chambers every two weeks to eliminate possible growth chamber effects.

Once chilling was completed, individuals were moved to a greenhouse with mean daytime temperature of 15$^{\circ}$C and a mean nighttime temperature of 10$^{\circ}$C. Photoperiod was set to 12 hour days throughout the spring until all individuals reached full leaf expansion. After all individuals of all species reached full leaf expansion, greenhouse temperatures and photoperiods were kept ambient. 

\subsection*{Phenology and False Spring Treatment}
Phenology observations were taken every 2-3 days through full leaf expansion and then recorded weekly over the summer. Budburst was denoted as BBCH stage 07, which is `beginning of sprouting or bud breaking' and monitored until full leaf expansion (BBCH stage 19) in order to evaluate the duration of vegetative risk \citep{Chamberlain2019} for each individual \citep{Finn2007}. Individuals in the `false spring treatment' were placed in a growth chamber set to mimic a false spring event during budburst, defined as once at least 50\% of the buds were at BBCH stage 07 but the individual had not yet reached BBCH stage 19 (that is, each individual was exposed to a false spring based on its individual phenological timing). Individuals receiving the false spring treatment were placed in a growth chamber for 14 hours, starting at 6pm. Temperatures in the growth chamber were ramped down over 14 hours (Figure \ref{fig:gccond}). After 8am the following day, individuals were placed back in the greenhouse with all of the other plants. Once all individuals reached full leaf expansion (BBCH stage 19), we made weekly phenology observations until August 1st and then we made observations every 2-3 days again to monitor fall phenology. Individuals were monitored until complete budset, at which point they were harvested for biomass measurements.

\subsection*{Growth measurements}
Growth was closely measured throughout the entirety of the experiment. Height was measured three times throughout the growing season: the day an individual reached full leaf expansion, 60 days after full leaf out and when an individual reached complete budset. We measured the chlorophyll content of four leaves on each individual 60 days after full leaf out using an atLEAF CHL PLUS Chlorophyll meter. The average chlorophyll content was calculated and then converted to mg/cm\textsuperscript{2} using the atLEAF CHL PLUS conversion tool. We measured leaf thickness using a Shars Digital Micrometer (scale works to 0.001mm) and leaf toughness in Newtons using a Shimpo Digital Force Gauge on two leaves for each individual. Additionally, we visually monitored damage to the shoot apical meristem, which consisted of complete damage or disruption of growth in the main stem and resulted in early dormancy induction or reliance on lateral shoot growth. Finally, belowground and aboveground biomass were harvested after an individual reached complete budset to include leaves in our biomass calculations. Belowground and aboveground plant material were separated and then put in a Shel Lab Forced Air Oven at 60$^{\circ}$C for at least 4 days. 

\subsection*{Data analysis}
Using Bayesian hierarchical models with the brms package \citep{brms}, version 2.3.1,  in R \citep{R}, version 3.3.1, we estimated the effects of chilling duration, false spring treatment and all two-way interactions as predictors on: (1) duration of vegetative risk, (2) growing season length, (3) shoot apical meristem damage, (4) total growth in centimeters, (5) total biomass, (6) chlorophyll content, (7) leaf toughness and (8) leaf thickness. Species were modeled hierarchically as grouping factors, which generated an estimate and posterior distribution of the overall response across the eight species used in our experiment. We ran four chains, each with 2 500 warm-up iterations and 4 000 sampling iterations for a total of 6 000 posterior samples for each predictor for each model using weakly informative priors. Increasing priors three-fold did not impact our results. We evaluated our model performance based on $\hat{R}$ values that were close to one and did not include models with divergent transitions in our results. We also evaluated high $n_{eff}$ (4000 for most parameters, but as low as 1400 for a couple of parameters in the shoot apical meristem model). We additionally assessed chain convergence and posterior predictive checks visually \citep{BDA}.


\section*{Results}
Chilling durations impacted individual phenology. As seen in many other studies, we found increases in chilling duration advanced day of budburst by -2.79 $\pm$ 1.74 days for six weeks of chilling and by -7.63 $\pm$ 1.76 days for eight weeks of chilling (Table \ref{tab:simpbb}). Longer chilling treatments also sped up leafout, reducing the duration of vegetative risk, especially in the eight weeks of chilling cohort (-2.67 $\pm$ 1.14 days; Figure \ref{fig:muphen}\textbf{a}, Table \ref{tab:simpdvr} and Table \ref{tab:suppmoddvr}). With increases in chilling duration, the growing season length decreased for individuals exposed to six and eight weeks of chilling (2.48 $\pm$ 4.87 days for six weeks and -9.66 $\pm$ 5 days for eight weeks; Figure \ref{fig:muphen}\textbf{b} and Table \ref{tab:suppmodgs}).
 
Additionally, false springs impacted individual phenology. Individuals exposed to the false spring treatment had longer durations of vegetative risk for the four weeks (2.97 $\pm$ 0.79 days) and slightly longer durations for the six weeks of chilling cohort (1.53 $\pm$ 1.14 days; Figure \ref{fig:muphen}\textbf{a} and Table \ref{tab:suppmoddvr}). Effects on the duration of vegetative risk from false spring (longer durations) and from increased chilling (shorter durations) were generally additive, resulting in no major changes in the durations of vegetative risk for individuals exposed to a false spring that received eight weeks of chilling (0.92 $\pm$ 1.08 days, Figure \ref{fig:muphen}\textbf{a} and Table \ref{tab:suppmoddvr}). 
  
False springs impacted growth habit and shoot growth but not total biomass. Across all chilling treatments, especially for the four and eight week cohorts, false springs led to more damage to the shoot apical meristem (51.7\% increase in probability of damage under false spring treatment or 2.07 $\pm$ 0.97 for four weeks, -5.2\% or a 1.33 $\pm$ 1.42 for six weeks and 7\% or a 2.17 $\pm$ 1.31 for eight weeks; Figure \ref{fig:mugrowth}\textbf{a} and Table \ref{tab:suppmodmeri}). Shoot growth over the growing season increased with eight weeks of chilling (11 $\pm$ 4.01 cm) except growth was not affected under false spring conditions (4.73 $\pm$ 5.49 cm; Figure \ref{fig:mugrowth}\textbf{b} and Table \ref{tab:suppmodht}). False springs led to slightly lower total biomasses when they were exposed to only four weeks of chilling (-3.45 $\pm$ 2.78 g) but there was very little change in total biomass under false spring conditions compared to the control for both the six weeks of chilling cohort (-3.62 $\pm$ 4.04 g) and the eight weeks of chilling cohort (2.88 $\pm$ 3.04 g; Figure \ref{fig:mutotbio} and Table \ref{tab:suppmodtotbio}).
  
False springs affected physical leaf traits. False springs led to decreased chlorophyll content in leaves with increases in chilling, especially with eight weeks of chilling (-1.45 $\pm$ 1.16 mg/cm\textsuperscript{2} for six weeks and -2.03 $\pm$ 1.07 mg/cm\textsuperscript{2} for eight weeks; Figure \ref{fig:muchl} and Table \ref{tab:suppmodchl}). False springs led to decreased leaf toughness across all chilling treatments (-0.05 $\pm$ 0.02 N for four weeks of chilling, -0.09 $\pm$ 0.03 N for six weeks of chilling and -0.08 $\pm$ 0.03 N for eight weeks of chilling; Figure \ref{fig:muleaf}\textbf{a} and Table \ref{tab:suppmodtough}). Additionally, false springs led to decreased leaf thickness across four and eight weeks of chilling cohorts, but there was little change for the six weeks of chilling cohort (-8.9 $\pm$ 3.74 $\mu$m for four weeks of chilling, -3.5 $\pm$ 5.31 $\mu$m for six weeks of chilling and -15.78 $\pm$ 5.25 $\mu$m for eight weeks of chilling; Figure \ref{fig:muleaf}\textbf{b} and Table \ref{tab:suppmodthick}).
  
False springs and chilling duration treatments resulted in only some species-level differences. Duration of vegetative risk decreased for most species with long chilling durations (i.e., the eight week cohort), except for \textit{Salix purpurea}, which experienced longer durations of vegetative risk with longer chilling durations (Figure \ref{fig:muphen}\textbf{a}). False springs led to meristem damage agross all species except for \textit{Betula populifolia} and \textit{Sorbus americana}. Additionally, \textit{Viburnum dentatum} experienced consistent meristem damage under all treatments (Figure \ref{fig:mugrowth}\textbf{a}). There was a lot of species-level variation with leaf thickness under the longer chilling durations, specifically with \textit{Sorbus americana} and \textit{Viburnum dentatum} having thicker leaves with increases in chilling (Figure \ref{fig:muleaf}\textbf{b}). 
  
Despite large treatment effects on phenology, we found no major effects on phenological rank within the community. Order of leafout timing was consistent across all treatments, with \textit{Salix purpurea} always being first to leafout, followed by \textit{Betula papyrifera}, \textit{B. populifolia} and \textit{Cornus racemosa} and finally by \textit{Alnus rugosa}, \textit{Sorbus americana}, \textit{Viburnum dentatum} and \textit{Acer saccharinum} (Figure \ref{fig:rank}). \textit{Viburnum dentatum} was the only species to change rank across treatments, though it was grouped consistently with the later-leafout group of species. Order of budset timing was also consistent across all treatments, with \textit{Cornus racemosa} and \textit{Sorbus americana} being first to set bud, followed by \textit{Betula papyrifera}, \textit{Acer saccharinum} and \textit{Viburnum dentatum}, and finally by \textit{B. populifolia}, \textit{Salix purpurea} and \textit{Alnus rugosa} (Figure \ref{fig:bsetrank}). \textit{Acer saccharinum} was the only species to change budset rank across treatments, though it was grouped consistently with \textit{Betula papyrifera} and \textit{Viburnum dentatum}.

\section*{Discussion} 
Our experiment allowed us to examine the cascading consequences of two major interactive effects of climate change across eight deciduous forest tree species---false springs and chilling. Our results confirmed the major features of false springs---causing plant damage---and chilling---advancing spring phenology---then highlighted how they altered multiple aspects of plant phenology, plant growth and leaf traits. Importantly, we found false springs and chilling have opposing additive effects on the duration of vegetative risk. This suggests that the combination of increased false springs and warmer winters could be especially detrimental to forest communities. 

\subsection*{False springs and chilling interactively determine risk and damage}
Chilling length greatly influences spring phenology during the critical budburst to leafout phases, and thus can compensate for the detrimental effects of false springs on phenology. With false springs increasing the duration of vegetative risk, the risk of multiple false springs occurring in one season also increases. But chilling can compensate for this increase in duration of vegetative risk: with increased chilling, the duration of vegetative risk does not increase under false spring conditions. This suggests chilling is more important for saplings in terms of exposure to multiple false springs. With climate change and warming temperatures, over-winter chilling is anticipated to decrease and false springs are predicted to increase in certain regions. This combination could greatly impact plant performance, survival and shape species distributions, ultimately affecting crucial processes such as carbon uptake and nutrient cycling.
  
False springs impact sapling growth, regardless of chilling duration. Past studies suggest early-budburst species can withstand lower temperature thresholds \citep{Lenz2013, Muffler2016, Zohner2020} but our results suggest false springs consistently impair shoot apical meristem growth, regardless of a species phenological order. Damage to the shoot apical meristem can lead to reliance on lateral shoot growth, rendering inefficient growth patterns and---if damage is significant within a stand---it can lead to declines in recruitment \citep{Rhodes2018}. Though overall height increased with more over-winter chilling, false springs impacted all individuals similarly across all chilling treatments, suggesting an interplay between false spring damage to the meristem and overall plant height and thus further emphasizing the detrimental effects of meristem damage. 
  
False springs greatly impact the physical characteristics of the leaf. With chlorophyll content, leaf toughness and leaf thickness decreasing, the quality of the leaf dwindled. This reduction in quality could subsequently lead to an increase in herbivory risk \citep{Onoda2011}. We found that increased chilling levels actually decreased leaf toughness and decreased chlorophyll content under false spring conditions. Further studies that assess the secondary compounds and total phenolic content \citep{Ayres1993, Webber2016} as well as photosynthetic rate of the leaves exposed to false springs after varying durations of chilling are needed to better understand the level of herbivory risk and the overall quality of the leaf. 

\subsection*{False springs and chilling do not reshape temporal assembly}
Climate model projections and little to no chilling treatments in experimental studies predict substantial shifts in species leafout order under climate change conditions \citep{Roberts2015, Laube2014}, other studies using long-term phenology observations suggest leafout phenology order is consistent across years \citep{Wesolowski2006}. We did not find major shifts in species leafout order---thus consistent with observational studies \citep{Wesolowski2006}---except for in \textit{Viburnum dentatum}, though it still leafed out within the later cohort of species across all treatments. Therefore, we do not predict major reassembly of forest communities due to winter warming or false spring incidence. As was evident from a full factorial growth chamber experiment, most treatments did not render substantial phenological reordering, except when individuals experienced little to no field chilling \citep{Laube2014}. Our results differ from projections---and experiments that lack chilling conditions---and this is likely due to greater decreases in over-winter chilling under future climate scenarios, which was not fully captured in our experimental design. % EMW21Jul: What do you mean by no-chill treatment? Are you trying to say that the conditions that induced ro-ordering in the Laube paper required no chill? We could say this more clearly when we write it up, it will just take more space, but seems an important point if that is what you mean % CJC 5 Aug: yes! Julia used a full factorial experiment and the only treatment where she saw reordering was under the little to no chilling field conditions. I tried to restructure the paragraph, let me know if it makes more sense!
    
Phenological rank remains consistent across all false spring and chilling treatments, as long as all species are affected equally. In nature, not all of our study species are at equal risk of false springs, with early-budburst species (e.g., \textit{Salix americana} or \textit{Betula papyrifera}) generally more at risk than later-budburst species (e.g., \textit{Acer saccharinum} or \textit{Viburnum dentatum}). This suggests we could in fact expect climate change to reshape forest communities, but it may not come from temporal assembly directly. Instead it might come from the damage and leaf trait impacts of false springs and decrease in chilling duration. Some temperate tree and shrub species utilize various leaf characteristics to dimish risk of false spring damage: increased `packability' of leaf primordia in winter buds which allows for more rapid leafout \citep{Edwards2017}, increased trichome density on young leaves to protect leaf tissue against freezes \citep{Agrawal2004, Prozherina2003} and buds with decreased water content to increase freeze tolerance \citep{Beck2007, Hofmann2015, Kathke2011, Morin2007,  Muffler2016, Nielsen2009, Poirier2010}. Thus our results suggest it will be the complex interplay of changing climate and species-specific mosaics of traits that will matter. 

Understanding forest recruitment and inter- and intraspecific competition with false springs is crucial. With over-winter chilling decreasing with climate change, saplings---which generally leafout earlier than later lifestages to gain access to light \citep{Augspurger2009}---are likely more at risk of sustaining damage from false spring events. %CJC 5 August: could we say "risk of exposure to false spring events." instead of "risk of sustaining damage from..."? I think this will lead into our caveat about early-budbursting species a bit better!
This could lead to dieback of saplings, most especially of early-budbursting species, in temperate forests with climate change. % EMW21Jul: Maybe -- but doesn't it depend on which species' seedlings experience the biggest CHANGE in false spring? It could be that early-spring species were always hit hard, so changes might more hit middle rank species' seedlings. See below for where I think you could sneak this in.
Thus, climate change could greatly impact early-budburst species, which will likely see increases in durations of vegetative risk with the dual effects of lower chilling and heightened false spring risk. Though this assumes that false springs affect early-budburst species due to their phenology and risk, but studies suggest earlier budbursting species generally can tolerate lower temperatures in the sping \citep{Lenz2013, Muffler2016}. Understanding how false springs are changing and how equally---or not---these effects are on different species and their seedlings is crucial for future projections. By integrating the additive and adverse effects of decreasing over-winter chilling and increasing false spring risk---and how false springs are changing across various species, we can better predict shifts in forest communities and recruitment under climate change. % EMW21Jul: One last question! Do you mean seedling or sapling? Should you define seedling somewhere quickly? My thought is that seedling can mean germinant to some people and I think 99% of those die, so it doesn't really matter if their false springs change a little for germinants. I'd just check and define your terminology somewhere (methods even would work). % CJC 5 August: You're right! I mean saplings!! I always think saplings need to be much older than a year. I changed seedling to sapling throughout the manuscript

\section*{Acknowledgments}

We would like to thank all of the growth facilities staff at the Weld Hill Research Building with a special thanks to Kea Woodruff for her continued dedication to the project and assistance in planting, experimental design and advice. We also want to thank Faye Rosin for her unceasing support in laboratory assistance and access as well as Wayne Daly for his invaluable dedication to the plants' health through the remainder of the experiece. We also thank D. Buonaiuto, A. Ettinger, J. Gersony, D. Loughnan, A. Manandhar and D. Sohdi for their continued feedback and insights that helped improved the experimental design, questions and manuscript.

\section*{Author Contribution}
C.J.C. and E.M.W. conceived aspects of the study and analysis and identified species to use in the study and determined which phenological and growth measurements to observe. C.J.C. performed the analyses and produced all figures and tables. Both authors contributed to the study design and edited the manuscript.

\section*{Data Availability:}
Data and code from the analyses will be available via KNB upon publication and are available to all reviewers upon request. Raw data, {Stan} model code and output are available on github at \url{https://github.com/cchambe12/chillfreeze} and provided upon request.


\bibliography{..//references/chillfreeze.bib}

\section*{Tables and Figures}

{\begin{figure} [H]
  -\begin{center}
  -\includegraphics[width=12cm]{..//analyses/figures/growthchamber.pdf}
  -\caption{False spring treament temperature regime in the growth chamber}\label{fig:gccond}
  -\end{center}
  -\end{figure}}
  
  {\begin{figure} [H]
  -\begin{center}
  -\includegraphics[width=18cm]{..//analyses/figures/mu_phen.pdf} 
  -\caption{Effects of false spring treatment, six weeks of chilling and eight weeks of chilling on a) duration of vegetative risk (days) and b) growing season length (days). Dots and lines show means and 90\% uncertainty intervals and thicker lines show 50\% uncertainty intervals.}\label{fig:muphen}
  -\end{center}
  -\end{figure}}
  
  {\begin{figure} [H]
  -\begin{center}
  -\includegraphics[width=18cm]{..//analyses/figures/mu_growth.pdf} 
  -\caption{Effects of false spring treatment, six weeks of chilling and eight weeks of chilling on a) shoot apical meristem damage and b) total shoot growth (cm). Dots and lines show means and 90\% uncertainty intervals and thicker lines show 50\% uncertainty intervals. }\label{fig:mugrowth}
  -\end{center}
  -\end{figure}}
  
  {\begin{figure} [H]
  -\begin{center}
  -\includegraphics[width=18cm]{..//analyses/figures/mu_leaftraits.pdf} 
  -\caption{Effects of false spring treatment, six weeks of chilling and eight weeks of chilling on a) leaf toughness (N) and b) leaf thickness ($\mu$m). Dots and lines show means and 90\% uncertainty intervals and thicker lines show 50\% uncertainty intervals. }\label{fig:muleaf}
  -\end{center}
  -\end{figure}}
  
  {\begin{figure} [H]
  -\begin{center}
  -\includegraphics[width=12cm]{..//analyses/figures/leafoutorder_byrank.pdf} 
  -\caption{Understanding rank order of leafout across all species using (a) mean trends and (b) raw estimates. }\label{fig:rank}
  -\end{center}
  -\end{figure}}
  

\end{document}
