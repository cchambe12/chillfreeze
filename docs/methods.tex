\documentclass{article}\usepackage[]{graphicx}\usepackage[]{color}
% maxwidth is the original width if it is less than linewidth
% otherwise use linewidth (to make sure the graphics do not exceed the margin)
\makeatletter
\def\maxwidth{ %
  \ifdim\Gin@nat@width>\linewidth
    \linewidth
  \else
    \Gin@nat@width
  \fi
}
\makeatother

\definecolor{fgcolor}{rgb}{0.345, 0.345, 0.345}
\newcommand{\hlnum}[1]{\textcolor[rgb]{0.686,0.059,0.569}{#1}}%
\newcommand{\hlstr}[1]{\textcolor[rgb]{0.192,0.494,0.8}{#1}}%
\newcommand{\hlcom}[1]{\textcolor[rgb]{0.678,0.584,0.686}{\textit{#1}}}%
\newcommand{\hlopt}[1]{\textcolor[rgb]{0,0,0}{#1}}%
\newcommand{\hlstd}[1]{\textcolor[rgb]{0.345,0.345,0.345}{#1}}%
\newcommand{\hlkwa}[1]{\textcolor[rgb]{0.161,0.373,0.58}{\textbf{#1}}}%
\newcommand{\hlkwb}[1]{\textcolor[rgb]{0.69,0.353,0.396}{#1}}%
\newcommand{\hlkwc}[1]{\textcolor[rgb]{0.333,0.667,0.333}{#1}}%
\newcommand{\hlkwd}[1]{\textcolor[rgb]{0.737,0.353,0.396}{\textbf{#1}}}%
\let\hlipl\hlkwb

\usepackage{framed}
\makeatletter
\newenvironment{kframe}{%
 \def\at@end@of@kframe{}%
 \ifinner\ifhmode%
  \def\at@end@of@kframe{\end{minipage}}%
  \begin{minipage}{\columnwidth}%
 \fi\fi%
 \def\FrameCommand##1{\hskip\@totalleftmargin \hskip-\fboxsep
 \colorbox{shadecolor}{##1}\hskip-\fboxsep
     % There is no \\@totalrightmargin, so:
     \hskip-\linewidth \hskip-\@totalleftmargin \hskip\columnwidth}%
 \MakeFramed {\advance\hsize-\width
   \@totalleftmargin\z@ \linewidth\hsize
   \@setminipage}}%
 {\par\unskip\endMakeFramed%
 \at@end@of@kframe}
\makeatother

\definecolor{shadecolor}{rgb}{.97, .97, .97}
\definecolor{messagecolor}{rgb}{0, 0, 0}
\definecolor{warningcolor}{rgb}{1, 0, 1}
\definecolor{errorcolor}{rgb}{1, 0, 0}
\newenvironment{knitrout}{}{} % an empty environment to be redefined in TeX

\usepackage{alltt}
\usepackage{Sweave}
\usepackage{float}
\usepackage{graphicx}
\usepackage{tabularx}
\usepackage{siunitx}
\usepackage{amssymb} % for math symbols
\usepackage{amsmath} % for aligning equations
\usepackage{textcomp}
\usepackage{mdframed}
\usepackage{natbib}
\bibliographystyle{..//references/styles/besjournals.bst}
\usepackage[small]{caption}
\setlength{\captionmargin}{30pt}
\setlength{\abovecaptionskip}{0pt}
\setlength{\belowcaptionskip}{10pt}
\topmargin -1.5cm        
\oddsidemargin -0.04cm   
\evensidemargin -0.04cm
\textwidth 16.59cm
\textheight 21.94cm 
%\pagestyle{empty} %comment if want page numbers
\parskip 7.2pt
\renewcommand{\baselinestretch}{1.5}
\parindent 0pt
%\usepackage{lineno}
%\linenumbers

%cross referencing:
\usepackage{xr}
%\usepackage{xr-hyper}
\externaldocument{/Users/CatherineChamberlain/Documents/git/chillfreeze/docs/chillfrz_outline}

\newmdenv[
  topline=true,
  bottomline=true,
  skipabove=\topsep,
  skipbelow=\topsep
]{siderules}
\IfFileExists{upquote.sty}{\usepackage{upquote}}{}
\begin{document}

\noindent \textbf{\Large{False spring damage on temperate tree seedlings is amplified with winter warming}}

\noindent Authors:\\
C. J. Chamberlain $^{1,2}$, K. Woodruff $^{1}$ \& E. M. Wolkovich $^{1,2,3}$
\vspace{2ex}\\
\emph{Author affiliations:}\\
$^{1}$Arnold Arboretum of Harvard University, 1300 Centre Street, Boston, Massachusetts, USA; \\
$^{2}$Organismic \& Evolutionary Biology, Harvard University, 26 Oxford Street, Cambridge, Massachusetts, USA; \\
$^{3}$Forest \& Conservation Sciences, Faculty of Forestry, University of British Columbia, 2424 Main Mall, Vancouver, BC V6T 1Z4\\
\vspace{2ex}
$^*$Corresponding author: 248.953.0189; cchamberlain@g.harvard.edu\\

\renewcommand{\thetable}{\arabic{table}}
\renewcommand{\thefigure}{\arabic{figure}}
\renewcommand{\labelitemi}{$-$}
\setkeys{Gin}{width=0.8\textwidth}

%%%%%%%%%%%%%%%%%%%%%%%%%%%%%%%%%%%%%%%%%%%%%%%
%%%%%%%%%%%%%%%%%%%%%%%%%%%%%%%%%%%%%%%%%%%%%%%
\section*{Methods}
\subsection*{Plant Selection and Material}
We used 8 temperate woody plant tree and shrub species with varying phenologies, that were not used as crops or ornamental species: \textit{Acer saccharinum} L., \textit{Alnus incana rugosa} L., \textit{Betula papyrifera} Marsh., \textit{Betula populifolia} Marsh., \textit{Cornus racemosa} Lam., \textit{Salix purpurea} L., \textit{Sorbus americana} Marsh., and \textit{Viburnum dentatum} L. Two additional species---\textit{Fagus grandifolia} and \textit{Nyssa sylvatica}---were originally included in the experimental design, but were not delivered in a usable condition and thus could not be used in the experiment. We received 48 dormant bare root seedlings---each measuring 6-12 inches---for each species from Cold Stream Farm LLC (Freesoil, MI; 44$^{\circ}$6' N -86$^{\circ}$12' W) for a total of 384 individuals. Upon receipt, plants were potted in POT INFO AND SOIL INFO HERE!! and placed in growth chambers at the Weld Hill Research Building of the Arnold Arboretum (Boston, MA; 42$^{\circ}$17' N -71$^{\circ}$8' W) at 4$^{\circ}$C to maintain dormancy. After all individuals had leafed out, all seedlings were up-potted to new pots (NEW POT SIZE HERE) and fertilized (FERTILIZER INFO HERE).

\subsection*{Growth Chamber and Greenhouse Conditions}
Individuals were randomly selected and placed in six experimental treatments: 4 weeks of chilling at 4$^{\circ}$C x no false spring, 4 weeks of chilling at 4$^{\circ}$C x false spring, 6 weeks of chilling at 4$^{\circ}$C x no false spring, 6 weeks of chilling at 4$^{\circ}$C x false spring,
8 weeks of chilling at 4$^{\circ}$C x no false spring, 8 weeks of chilling at 4$^{\circ}$C x false spring. While individuals were in the growth chamber under chilling conditions, photoperiod was maintained at eight hour days. Lighting within the chambers was provided through a combination of T5HO fluorescent lamps with halogen incandescent bulbs at roughly 250 $\mu mol/m^{2}/s$. Individuals were rotated within and among growth chambers every two weeks to eliminate possible growth chamber effects.

Once chilling was completed, individuals were moved to a greenhouse with mean daytime temperature of 15$^{\circ}$C and a mean nighttime temperature of 10$^{\circ}$C. Photoperiod was set to 12 hour days throughout the spring until all individuals reached full leaf expansion. After all individuals of all species reached full leaf expansion, greenhouse temperatures and photoperiods were kept ambient. 

\subsection*{Phenology and False Spring Treatment}
Phenology observations were taken every 2-3 days through full leaf expansion and then recorded weekly over the summer. Budburst was denoted as BBCH stage 07, which is `beginning of sprouting or bud breaking' and monitored until full leaf expansion (BBCH stage 19) in order to evaluate the duration of vegetative risk \citep{Chamberlain2019} for each individual \citep{Finn2007}. Individuals in the `false spring treatment' were placed in a growth chamber set to mimic a false spring event during budburst, defined as once at least 50\% of the buds were at BBCH stage 07 but the individual had not yet reached BBCH stage 19 (that is, each individual was exposed to a false spring based on its individual phenological timing). Individuals receiving the false spring treatment were placed in a growth chamber for 14 hours, starting at 6pm. Temperatures in the growth chamber were ramped down over 14 hours (Figure \ref{fig:gccond}). After 8am the following day, individuals were placed back in the greenhouse with all of the other plants. Once all individuals reached full leaf expansion (BBCH stage 19), we made weekly phenology observations until August 1st and then we made observations every 2-3 days again to monitor fall phenology. Individuals were monitored until complete budset, at which point they were harvested for biomass measurements.

\subsection*{Growth measurements}
Growth was closely measured throughout the entirety of the experiment. Height was measured three times throughout the growing season: the day an individual reached full leaf expansion, 60 days after full leaf out and when an individual reached complete budset. We measured the chlorophyll content of four leaves on each individual 60 days after full leaf out using an atLEAF CHL PLUS Chlorophyll meter. The average chlorophyll content was calculated and then converted to mg/cm\textsuperscript{2} using the atLEAF CHL PLUS conversion tool. We measured leaf thickness using a Shars Digital Micrometer (scale works to 0.001mm) and leaf toughness in Newtons using a Shimpo Digital Force Gauge on two leaves for each individual. Additionally, we visually monitored damage to the shoot apical meristem, which consisted of complete damage or disruption of growth in the main stem and resulted in early dormancy induction or reliance on lateral shoot growth. Finally, belowground and aboveground biomass were harvested after an individual reached complete budset to include leaves in our biomass calculations. Belowground and aboveground plant material were separated and then put in a Shel Lab Forced Air Oven at 60$^{\circ}$C for at least 4 days. 

\subsection*{Data analysis}
Using Bayesian hierarchical models with the brms package \citep{brms}, version 2.3.1,  in R \citep{R}, version 3.3.1, we estimated the effects of chilling duration, false spring treatment and all two-way interactions as predictors on: (1) duration of vegetative risk, (2) growing season length, (3) total growth in centimeters, (4) chlorophyll content, (5) leaf thickness, (6) leaf toughness, (7) shoot apical meristem damage and (8) total biomass. Species were modeled hierarchically as grouping factors, which generated an estimate and posterior distribution of the overall response across the eight species used in our experiment. We ran four chains, each with 2 500 warm-up iterations and 4 000 sampling iterations for a total of 6 000 posterior samples for each predictor for each model using weakly informative priors. Increasing priors three-fold did not impact our results. We evaluated our model performance based on $\hat{R}$ values that were close to one and did not include models with divergent transitions in our results. We also evaluated high $n_{eff}$ (4000 for most parameters, but as low as 1400 for a couple of parameters in the shoot apical meristem model). We additionally assessed chain convergence and posterior predictive checks visually \citep{BDA}.



\bibliography{..//references/chillfreeze.bib}

\end{document}
