\documentclass{article}\usepackage[]{graphicx}\usepackage[]{color}
% maxwidth is the original width if it is less than linewidth
% otherwise use linewidth (to make sure the graphics do not exceed the margin)
\makeatletter
\def\maxwidth{ %
  \ifdim\Gin@nat@width>\linewidth
    \linewidth
  \else
    \Gin@nat@width
  \fi
}
\makeatother

\definecolor{fgcolor}{rgb}{0.345, 0.345, 0.345}
\newcommand{\hlnum}[1]{\textcolor[rgb]{0.686,0.059,0.569}{#1}}%
\newcommand{\hlstr}[1]{\textcolor[rgb]{0.192,0.494,0.8}{#1}}%
\newcommand{\hlcom}[1]{\textcolor[rgb]{0.678,0.584,0.686}{\textit{#1}}}%
\newcommand{\hlopt}[1]{\textcolor[rgb]{0,0,0}{#1}}%
\newcommand{\hlstd}[1]{\textcolor[rgb]{0.345,0.345,0.345}{#1}}%
\newcommand{\hlkwa}[1]{\textcolor[rgb]{0.161,0.373,0.58}{\textbf{#1}}}%
\newcommand{\hlkwb}[1]{\textcolor[rgb]{0.69,0.353,0.396}{#1}}%
\newcommand{\hlkwc}[1]{\textcolor[rgb]{0.333,0.667,0.333}{#1}}%
\newcommand{\hlkwd}[1]{\textcolor[rgb]{0.737,0.353,0.396}{\textbf{#1}}}%
\let\hlipl\hlkwb

\usepackage{framed}
\makeatletter
\newenvironment{kframe}{%
 \def\at@end@of@kframe{}%
 \ifinner\ifhmode%
  \def\at@end@of@kframe{\end{minipage}}%
  \begin{minipage}{\columnwidth}%
 \fi\fi%
 \def\FrameCommand##1{\hskip\@totalleftmargin \hskip-\fboxsep
 \colorbox{shadecolor}{##1}\hskip-\fboxsep
     % There is no \\@totalrightmargin, so:
     \hskip-\linewidth \hskip-\@totalleftmargin \hskip\columnwidth}%
 \MakeFramed {\advance\hsize-\width
   \@totalleftmargin\z@ \linewidth\hsize
   \@setminipage}}%
 {\par\unskip\endMakeFramed%
 \at@end@of@kframe}
\makeatother

\definecolor{shadecolor}{rgb}{.97, .97, .97}
\definecolor{messagecolor}{rgb}{0, 0, 0}
\definecolor{warningcolor}{rgb}{1, 0, 1}
\definecolor{errorcolor}{rgb}{1, 0, 0}
\newenvironment{knitrout}{}{} % an empty environment to be redefined in TeX

\usepackage{alltt}
\usepackage{Sweave}
\usepackage{float}
\usepackage{graphicx}
\usepackage{tabularx}
\usepackage{siunitx}
\usepackage{mdframed}
\usepackage{natbib}
\bibliographystyle{..//papers/styles/besjournals.bst}
\usepackage[small]{caption}
\setkeys{Gin}{width=0.8\textwidth}
\setlength{\captionmargin}{30pt}
\setlength{\abovecaptionskip}{0pt}
\setlength{\belowcaptionskip}{10pt}
\topmargin -1.5cm        
\oddsidemargin -0.015cm   
\evensidemargin -0.015cm
\textwidth 16cm
\textheight 21cm 
%\pagestyle{empty} %comment if want page numbers
\parskip 7.2pt
\renewcommand{\baselinestretch}{2}
\parindent 20pt
\usepackage{indentfirst} 

\newmdenv[
  topline=true,
  bottomline=true,
  skipabove=\topsep,
  skipbelow=\topsep
]{siderules}
\IfFileExists{upquote.sty}{\usepackage{upquote}}{}
\begin{document}

\renewcommand{\thetable}{\arabic{table}}
\renewcommand{\thefigure}{\arabic{figure}}
\renewcommand{\labelitemi}{$-$}

\renewcommand{\thesection}{\arabic{section}.}
\renewcommand\thesubsection{\arabic{section}.\arabic{subsection}} 

%%%%%%%%%%%%%%%%%%%%%%%%%%%%%%%%%%%%
\section*{Experiment Breakdown}

Experiment Part 1 - Recieved bare root seedlings from Cold Stream Farm. Potted on Monday, 26 November 2018. Three cohorts for chilling lengths. I have two growth chambers, both at 4$^{\circ}$C. I will rotate seedlings every two weeks. After 4 weeks, I will remove the first cohort and move into the greenhouse. I will remove the second cohort at 6 weeks and final cohort at 8 weeks. Experiment Part 2 - I will monitor the phenology for the each plant. I will observe budburst (BBCH 09) until full leafout (BBCH 19) for each plant and note floral development - if present. Experiment Part 3 - once at least 50\% of the buds have reached BBCH 9, I will put the seedling in a growth chamber and ramp the temperature from the greenhouse temperature down to  -3$^{\circ}$C and then ramp up again after 3 hours at -3 for a 24 hour cycle. I will then move the seedlings back into the greenhouse to monitor percent budburst and record damage.

\subsection*{Finer details}
\begin{itemize}
\item[$\bullet$] Chilling conditions: 8 hour photoperiod and constant 4C
\item[$\bullet$] Greenhouse conditions: 12 hour photoperiod, 15C during the day and 10C at night, ambient humidity
\item[$\bullet$] Growth Chamber: Ramp down temperature starting at 6pm to 10C, 8pm to 5C, 10pm to 0C, 12am to -3C, 3am to 0C, 4am to 5C, 6am to 10C and 8am to 15C. 12 hour photoperiod (6-6) 
\end{itemize}

\newpage
\begin{center}
\captionof{table}{List of species being used in experiment. 10 species, 16 per chilling treatment and 8 per false spring treatment per chilling treatment.}\label{tab:exp1} 
\footnotesize
\begin{tabular}{|c | c | c | c |}
\hline
\textbf{Species} & \textbf{No. of Individs} \\
\hline
FAGGRA (NA) & 48  \\
\hline
BETPAP & 48 \\
\hline
BETPOP & 48 \\
\hline
SORAME & 48 \\
\hline
ALNRUG & 48 \\
\hline
NYSSYL (NA) & 48 \\
\hline
ACESAC & 48 \\
\hline
VIBDEN & 48 \\
\hline
SALPUR & 48 \\
\hline
CORRAC & 48 \\
\hline
\end{tabular}
\end{center}

\end{document}
