\documentclass[11pt,a4paper]{article}\usepackage[]{graphicx}\usepackage[]{color}
% maxwidth is the original width if it is less than linewidth
% otherwise use linewidth (to make sure the graphics do not exceed the margin)
\makeatletter
\def\maxwidth{ %
  \ifdim\Gin@nat@width>\linewidth
    \linewidth
  \else
    \Gin@nat@width
  \fi
}
\makeatother

\definecolor{fgcolor}{rgb}{0.345, 0.345, 0.345}
\newcommand{\hlnum}[1]{\textcolor[rgb]{0.686,0.059,0.569}{#1}}%
\newcommand{\hlstr}[1]{\textcolor[rgb]{0.192,0.494,0.8}{#1}}%
\newcommand{\hlcom}[1]{\textcolor[rgb]{0.678,0.584,0.686}{\textit{#1}}}%
\newcommand{\hlopt}[1]{\textcolor[rgb]{0,0,0}{#1}}%
\newcommand{\hlstd}[1]{\textcolor[rgb]{0.345,0.345,0.345}{#1}}%
\newcommand{\hlkwa}[1]{\textcolor[rgb]{0.161,0.373,0.58}{\textbf{#1}}}%
\newcommand{\hlkwb}[1]{\textcolor[rgb]{0.69,0.353,0.396}{#1}}%
\newcommand{\hlkwc}[1]{\textcolor[rgb]{0.333,0.667,0.333}{#1}}%
\newcommand{\hlkwd}[1]{\textcolor[rgb]{0.737,0.353,0.396}{\textbf{#1}}}%
\let\hlipl\hlkwb

\usepackage{framed}
\makeatletter
\newenvironment{kframe}{%
 \def\at@end@of@kframe{}%
 \ifinner\ifhmode%
  \def\at@end@of@kframe{\end{minipage}}%
  \begin{minipage}{\columnwidth}%
 \fi\fi%
 \def\FrameCommand##1{\hskip\@totalleftmargin \hskip-\fboxsep
 \colorbox{shadecolor}{##1}\hskip-\fboxsep
     % There is no \\@totalrightmargin, so:
     \hskip-\linewidth \hskip-\@totalleftmargin \hskip\columnwidth}%
 \MakeFramed {\advance\hsize-\width
   \@totalleftmargin\z@ \linewidth\hsize
   \@setminipage}}%
 {\par\unskip\endMakeFramed%
 \at@end@of@kframe}
\makeatother

\definecolor{shadecolor}{rgb}{.97, .97, .97}
\definecolor{messagecolor}{rgb}{0, 0, 0}
\definecolor{warningcolor}{rgb}{1, 0, 1}
\definecolor{errorcolor}{rgb}{1, 0, 0}
\newenvironment{knitrout}{}{} % an empty environment to be redefined in TeX

\usepackage{alltt}
\usepackage[top=1.00in, bottom=1.0in, left=1.1in, right=1.1in]{geometry}
\usepackage{graphicx}
\usepackage[numbers]{natbib}
\bibliographystyle{..//..//references/styles/nature.bst}

\usepackage[export]{adjustbox}
\IfFileExists{upquote.sty}{\usepackage{upquote}}{}
\begin{document}

\includegraphics[width=0.5\textwidth, right]{AA_logo.jpg}
\noindent 1300 Centre Street\\
\noindent Boston, MA, 20131\\

\vspace{1.5ex}

\pagenumbering{gobble}

\noindent{Dear Dr. Gibson:}
\vspace{3ex}\\
\noindent Please consider our manuscript, `False spring damage to temperate tree saplings is amplified with winter warming', as a Research Article for the \textit{Journal of Ecology}. \\

\noindent Biological spring is advancing with climate change \citep{Polgar2014,Fu2015} but late spring freeze dates are not predicted to advance at the same rate \citep{Labe2016,Sgubin2018}, leading to renewed interest in late spring freeze events---commonly called `false springs'---which shape the life history of many temperate and boreal plant species. Additionally, over-winter chilling is predicted to decrease with warming winters, which could impact plant phenology and, ultimately, growth. \\

\noindent Here, we dissect the interplay of false springs coupled with warming winters across eight temperate deciduous tree species to better understand how our forests will be reshaped under climate change. We found that false springs increased the rate of budburst, increased damage to the shoot apical meristem, and decreased leaf toughness, leaf thickness and chlorophyll content but did not cause phenological reordering within a community. Longer chilling led to decreased rates of budburst, even under false spring conditions, thus chilling compensated for the adverse effects of false springs on phenology. We therefore expect climate change to reshape forest communities not through temporal reassembly but rather through impacts on growth and leaf traits from the coupled effects of false springs with decreases in over-winter chilling under future climate change scenarios.\\

\noindent We believe our research is important and timely because our findings suggest climate change could diminish or reverse the positive effects of carbon storage in temperate forests. Damage from false spring events with decreases in over-winter chilling may have cascading effects to pollinators \citep{Boggs2012,Pardee2017}, nutrient cycling and carbon uptake as well as forest recruitment \citep{Hufkens2012,Klosterman2018,Richardson2013}. \\

\noindent This manuscript is not under consideration elsewhere. Both authors approved of this version for submission. We hope that you will find it suitable for publication in the \textit{Journal of Ecology}. Thank you for your consideration. \\

\vspace{1.5ex}
\noindent Sincerely, \\
\includegraphics[width=0.2\textwidth]{full_signature.jpg} \\
\noindent Catherine Chamberlain (on behalf of my co-authors)
\vspace{2ex}\\
\noindent Authors:\\
C. J. Chamberlain $^{1,2}$ \& E. M. Wolkovich $^{1,2,3}$
\vspace{2ex}\\
\emph{Author affiliations:}\\
$^{1}$Arnold Arboretum of Harvard University, 1300 Centre Street, Boston, Massachusetts, USA; \\
$^{2}$Organismic \& Evolutionary Biology, Harvard University, 26 Oxford Street, Cambridge, Massachusetts, USA; \\
$^{3}$Forest \& Conservation Sciences, Faculty of Forestry, University of British Columbia, 2424 Main Mall, Vancouver, BC V6T 1Z4\\
\vspace{2ex}
$^*$Corresponding author: 248.953.0189; cchamberlain@g.harvard.edu\\

\bibliography{..//..//references/chillfreeze.bib}

\end{document}
