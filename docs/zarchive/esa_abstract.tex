\documentclass{article}\usepackage[]{graphicx}\usepackage[]{color}
% maxwidth is the original width if it is less than linewidth
% otherwise use linewidth (to make sure the graphics do not exceed the margin)
\makeatletter
\def\maxwidth{ %
  \ifdim\Gin@nat@width>\linewidth
    \linewidth
  \else
    \Gin@nat@width
  \fi
}
\makeatother

\definecolor{fgcolor}{rgb}{0.345, 0.345, 0.345}
\newcommand{\hlnum}[1]{\textcolor[rgb]{0.686,0.059,0.569}{#1}}%
\newcommand{\hlstr}[1]{\textcolor[rgb]{0.192,0.494,0.8}{#1}}%
\newcommand{\hlcom}[1]{\textcolor[rgb]{0.678,0.584,0.686}{\textit{#1}}}%
\newcommand{\hlopt}[1]{\textcolor[rgb]{0,0,0}{#1}}%
\newcommand{\hlstd}[1]{\textcolor[rgb]{0.345,0.345,0.345}{#1}}%
\newcommand{\hlkwa}[1]{\textcolor[rgb]{0.161,0.373,0.58}{\textbf{#1}}}%
\newcommand{\hlkwb}[1]{\textcolor[rgb]{0.69,0.353,0.396}{#1}}%
\newcommand{\hlkwc}[1]{\textcolor[rgb]{0.333,0.667,0.333}{#1}}%
\newcommand{\hlkwd}[1]{\textcolor[rgb]{0.737,0.353,0.396}{\textbf{#1}}}%
\let\hlipl\hlkwb

\usepackage{framed}
\makeatletter
\newenvironment{kframe}{%
 \def\at@end@of@kframe{}%
 \ifinner\ifhmode%
  \def\at@end@of@kframe{\end{minipage}}%
  \begin{minipage}{\columnwidth}%
 \fi\fi%
 \def\FrameCommand##1{\hskip\@totalleftmargin \hskip-\fboxsep
 \colorbox{shadecolor}{##1}\hskip-\fboxsep
     % There is no \\@totalrightmargin, so:
     \hskip-\linewidth \hskip-\@totalleftmargin \hskip\columnwidth}%
 \MakeFramed {\advance\hsize-\width
   \@totalleftmargin\z@ \linewidth\hsize
   \@setminipage}}%
 {\par\unskip\endMakeFramed%
 \at@end@of@kframe}
\makeatother

\definecolor{shadecolor}{rgb}{.97, .97, .97}
\definecolor{messagecolor}{rgb}{0, 0, 0}
\definecolor{warningcolor}{rgb}{1, 0, 1}
\definecolor{errorcolor}{rgb}{1, 0, 0}
\newenvironment{knitrout}{}{} % an empty environment to be redefined in TeX

\usepackage{alltt}[12pt]
\usepackage{Sweave}
\usepackage{float}
\usepackage{graphicx}
\usepackage{tabularx}
\usepackage{siunitx}
\usepackage{amssymb} % for math symbols
\usepackage{amsmath} % for aligning equations
\usepackage{mdframed}
\usepackage{natbib}
\bibliographystyle{..//bib/styles/gcb}
\usepackage[hyphens]{url}
\usepackage[small]{caption}
\setlength{\captionmargin}{30pt}
\setlength{\abovecaptionskip}{0pt}
\setlength{\belowcaptionskip}{10pt}
\topmargin -1.5cm        
\oddsidemargin -0.04cm   
\evensidemargin -0.04cm
\textwidth 16.59cm
\textheight 21.94cm 
%\pagestyle{empty} %comment if want page numbers
\parskip 7.2pt
\renewcommand{\baselinestretch}{2}
\parindent 0pt
\usepackage{lineno}
\linenumbers
\usepackage{setspace}
\doublespacing

\newmdenv[
  topline=true,
  bottomline=true,
  skipabove=\topsep,
  skipbelow=\topsep
]{siderules}
\IfFileExists{upquote.sty}{\usepackage{upquote}}{}
\begin{document}


\noindent \textbf{\Large{False spring damage on temperate tree seedlings is amplified with winter warming}}

\noindent Authors:\\
C. J. Chamberlain $^{1,2}$, K. Woodruff $^{1}$ \& E. M. Wolkovich $^{1,2,3}$
\vspace{2ex}\\
\emph{Author affiliations:}\\
$^{1}$Arnold Arboretum of Harvard University, 1300 Centre Street, Boston, Massachusetts, USA; \\
$^{2}$Organismic \& Evolutionary Biology, Harvard University, 26 Oxford Street, Cambridge, Massachusetts, USA; \\
$^{3}$Forest \& Conservation Sciences, Faculty of Forestry, University of British Columbia, 2424 Main Mall, Vancouver, BC V6T 1Z4\\
\vspace{2ex}
$^*$Corresponding author: 248.953.0189; cchamberlain@g.harvard.edu\\

\renewcommand{\thetable}{\arabic{table}}
\renewcommand{\thefigure}{\arabic{figure}}
\renewcommand{\labelitemi}{$-$}
\setkeys{Gin}{width=0.8\textwidth}

%%%%%%%%%%%%%%%%%%%%%%%%%%%%%%%%%%%%%%%%%%%%%%%
%%%%%%%%%%%%%%%%%%%%%%%%%%%%%%%%%%%%%%%%%%%%%%%

With warming temperatures in the Northern Hemisphere, spring phenology (i.e., budburst and leafout) is advancing. Temperate plants are at risk of freezing temperatures and have evolved to minimize risk through myriad strategies, with the most effective being avoidance: plants must exhibit flexible spring phenologies in order to maximize growth and minimize spring freeze risk by timing budburst effectively. Late spring freezing events that occur after trees initiate budburst are known as false springs and they are predicted to increased in intensity in certain regions as climate change progresses. Budburst in trees and shrubs requires three cues: (1) over-winter cold temperatures (chilling), (2) warming spring temeratures (forcing) and (3) longer daylengths (photoperiod). With climate change advancing, chilling is predicted to decrease as winter temperatures warm, potentially impacting phenology and, ultimately, growth. We assessed the effects of three levels of over-winter chilling on seedling phenology and growth across ten temperate tree and shrub species. Once budburst initiated, half of the individuals were exposed to a false spring treatment at -3$^{\circ}$C for three hours in a controlled growth chamber environment. Individuals remained in a greenhouse for the rest of the growing season where phenology and growth traits were measured and monitored.

We found that false springs impacted growth and reduced the length of the growing season. Individuals exposed to the false spring treatment took longer to fully leafout at the beginning of the season and individuals initiated dormancy earlier at the end of the season, thus reducing the length of the growing season. Additionally, across all species and chilling treatments, individuals exposed to false springs had more damage to the main shoot and relied more on lateral shoot growth, rendering inefficient growth patterns. However, individuals grew more with increased over-winter chilling. With sufficient chilling, growth and biomass were much higher---whether or not they were exposed to a false spring---suggesting chilling is more important for seedling growth than avoiding false springs. With climate change and warming temperatures, over-winter chilling is anticipated to decrease and false springs are predicted to increase in certain regions. This combination could greatly impact plant performance, survival and shape species distributions, ultimately affecting crucial processes such as carbon uptake and nutrient cycling.



\end{document}
